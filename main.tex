\documentclass[11pt]{article} % Default document font size

\usepackage{ctex}

\usepackage{enumitem}
\setitemize[1]{itemsep=0em,parsep=0em,topsep=0em,partopsep=0em}

\usepackage{graphicx} % Required for including images

\setlength{\parindent}{0pt} % Stop paragraph indentation

\usepackage{geometry} % Required for adjusting page dimensions and margins

\geometry{
	paper=a4paper, % Paper size, change to letterpaper for US letter size
	top=2.0cm, % Top margin
	bottom=2.0cm, % Bottom margin
	left=2.0cm, % Left margin
	right=2.0cm, % Right margin
	headheight=0.75cm, % Header height
	footskip=1cm, % Space from the bottom margin to the baseline of the footer
	headsep=0.75cm, % Space from the top margin to the baseline of the header
	%showframe, % Uncomment to show how the type block is set on the page
}

\usepackage[utf8]{inputenc} % Required for inputting international characters

\usepackage{sectsty} % Allows changing the font options for sections in a document

\usepackage[usenames, dvipsnames]{xcolor} % Required for specifying colours by name

\usepackage[bookmarks, colorlinks, breaklinks]{hyperref} % Required for links

% Set link colours
\hypersetup{
	linkcolor=blue,
	citecolor=blue,
	filecolor=black,
	urlcolor=cyan,  % MidnightBlue
}

\usepackage{ulem}

\renewcommand{\baselinestretch}{1.1}


% Set PDF meta-information

\hypersetup{
    pdftitle={Zekun Lou - Curriculum vitae},
    pdfauthor={Zekun Lou}
    }

\begin{document}

\begin{center}

\textbf{\Large Zekun Lou ({\kaishu 娄泽坤})}

Department of Physics, Fudan University

No. 220 Handan Road, Shanghai 200433, China

{\itshape Phone}: +86-18738402676 \
{\itshape Email}: \href{mailto:zklou19@fudan.edu.cn}{zklou19@fudan.edu.cn}
\end{center}

\vspace{-3ex}
\rule{\textwidth}{1pt}
\vspace{-4ex}
%%%%%%%%%%%%%%%%%%%%%%%%%%%%%%%%%%%%%%%%%%%%%%%%%%%%%%%%%%



\section*{Education}

B.S. in Physics, Fudan University, China
\hfill 09/2019 - 07/2023 (Expected)

\begin{itemize}
    \item Cumulative GPA: 3.80/4.0 (Department rank: 6/104).
    \item Selected into Honored Student Program (top 10\% admitted) each year.
    \item Relevant courses: Computational Physics, Solid State Physics, (Advanced) Quantum Mechanics, (Advanced) Electrodynamics, \verb|C++/Python| Programming, Machine Learning (via Cousera).
\end{itemize}

% Pre-college Education, in Henan, China
% \hfill Born - 07/2019



% \section*{\sout{Publications} {\tiny Not yet. Wait and see.}}

\section*{Research Interests}

Artificial Intelligence for Science (Materials Science, Computational Physics, Chemistry), Many-body Interaction (excitons and coupling)


\section*{Research Projects}

\textbf{Neural network XC functionals}
\hfill Shanghai AI Laboratory

Advisor: \href{https://scholar.google.com/citations?user=X_ZfX8sAAAAJ}{Dr. Han-Sen Zhong}
\hfill 09/2022 - Present

\begin{itemize}
    \item Develop neural network-based exchange-correlation functionals ($V_{\mathrm{XC}}^{\mathrm{NN}}$) for Kohn-Sham density functional theory.
    \item Set up datasets consists of $V_{\mathrm{XC}}$ with $\rho$ labels of different chemical accuracies by PySCF and Gaussian.
    \item Build Unet-inspired $V_{\mathrm{XC}}^{\mathrm{NN}}$ on rasterized grids and quadrature grids. Build equivariant graph convolutional $V_{\mathrm{XC}}^{\mathrm{NN}}$ with atom-centered electron descriptors. Deploy above $V_{\mathrm{XC}}^{\mathrm{NN}}$s in PySCF self-consistent field calculation.
\end{itemize}



\textbf{Reversible bialloy descriptor}
\hfill Fudan University

Advisor: \href{https://scholar.google.com/citations?user=5GcATiIAAAAJ}{Prof. Hongjun Xiang}
\hfill 02/2022 - Present

\begin{itemize}
    \item Search for "good" descriptors of bialloy structures. Perform  reconstruction tests on Fourier transformation coefficients and cluster expansion messages.
    \item Encode bialloy structures by Graph Neural Networks and build generative models by Variational Autoencoder for property optimization.
\end{itemize}



\textbf{Pirani gauge teaching experiment}
\hfill Fudan University

Advisor: \href{http://phylab.fudan.edu.cn/doku.php?id=home:xiaole}{Prof. Yongkang Le}
\hfill 06/2021 - 08/2022

\begin{itemize}
    \item Develop a Pirani gauge teaching experiment.
    \item Everything starts from zero, including circuit designing, microcontroller programming, and PID tuning. Learned lots of techniques and lab skills. User instructions and lab handouts are available for undergraduate students.
\end{itemize}


\textbf{Vacuum chamber and puck design}
\hfill Fudan University

Advisor: \href{https://scholar.google.com/citations?user=IdLAVPsAAAAJ}{Prof. Yuanbo Zhang}
\hfill 09/2021 - 03/2022

\begin{itemize}
    \item Design sample pucks and functional vacuum chambers for an ultra-low temperature \& strong magnetic field dilution refrigerator.
    \item Design a tricky and robust mechanism for installing/removing the puck from the cold dock plate.
\end{itemize}




\section*{Course Projects}

\textbf{Invisible cloak}
\hfill 09/2021 - 12/2021

{\itshape Electrodynamics I}. Advised by \href{https://scholar.google.com/citations?user=4x9SoV0AAAAJ}{Prof. Lei Zhou}.

\begin{itemize}
    \item Design electromagnetic field controlling meta-structures based on transformation optics and conformal mapping. Analyse energy scattering caused by impedance mismatch, and correct it using gradient-index meta-surfaces.
    \item Verify the above theoretical results by COMSOL FEA simulation.
\end{itemize}


\textbf{Heisenberg model in nano particles}
\hfill 03/2021 - 06/2021

{\itshape Statistical Physics I}, best two in all 13 projects.

\begin{itemize}
    \item Research the behaviors of anisotropic Heisenberg model in nanomagnetic particles.
    \item Implement Monte Carlo Metropolis algorithm in \verb|C++|.
\end{itemize}

\textbf{Drifting Speckles in laser spots}
\hfill 09/2019 - 08/2020

{\itshape Basic Physics Modeling}, best two in all 17 projects.

\begin{itemize}
    \item Research the properties of the laser speckle phenomenon, mainly about Fourier optics.
    \item Conducted experiments, theoretical analysis, and computational simulations (in \verb|C| and \verb|OpenMP|).
\end{itemize}



\section*{Skills}

\textbf{Second language}: English. IELTS 7.0. TOEFL iBT 100 (Speaking 24).

\textbf{Programming Languages}:
Python (PyTorch, PySCF, NumPy, Pandas, etc.), Wolfram Language (\href{https://www.bilibili.com/video/BV1uA4y1X7tU/}{my lecture video}), \verb|C/C++|, \LaTeX, PowerShell, bash, \verb|C/microPython| for SCMs.

\textbf{Software}:
VS Code, Mathematica, COMSOL, LAMMPS, VESTA, SOLIDWORKS, Multisim, Arduino, Keil5, Origin, Mathcha Notebook, MikTeX, Overleaf, MS Office.



\section*{Honors \& Prizes}

National undergraduate scholarship (top scholarship, top 2\% in Fudan)
\hfill 2020, 2022

First class scholarship of Fudan University (top 5\% in Fudan)
\hfill 2021

Freshmen scholarship of Fudan University (for outstanding in CPhO 2018)
\hfill 2019

\vspace{2ex}

National top student project in basic science (top 10\% in physics)
\hfill 2020, 2021, 2022

Honored student in physics department (top 10\% in physics)
\hfill 2020, 2021, 2022

\vspace{2ex}

First prize in China Undergraduate Mathematical Contest in Modeling (Shanghai)
\hfill 2022

First prize in National Physics Competition for College Students
\hfill 2019, 2021

First prize in National Physics Experiment Competition for College Students  (rank first)
\hfill 2021

Second prize in China Undergraduate Physics Tournament (CUPT)
\hfill 2020

Second prize in China Physics Olympiad (CPhO 2018, in high school)
\hfill 2018



% \section*{Interests}

% \textbf{Academic}:

% AI for science (epsecially in materials science and computational chemistry), solid state physics, quantum computing.

% \textbf{Sports \& Musical}:
% Badminton, swimming. Accordion.


\end{document}
